\section{INTRODUCTION}
\IEEEPARstart{F}{requency} is an important parameter in power systems. Accurate frequency estimation is imperative from the point of view of power systems protection. Various methods have been developed up till now for estimation of frequency and all have their distinct advantages. The modified zero-crossing technique \cite{modzcd}, level crossing technique \cite{levelcross}, least-square technique \cite{leastsquare}, Kalman filter technique \cite{kalman1},\cite{kalman2}, and phasor based techniques \cite{leakeffect,phasor1,lovell,synchro,phadkebook} are fixed-frequency methods suitable for steady-state frequency measurement. Methods utilizing feedback loops such as \cite{phasecontrol}; Phase Locked Loop(PLL); phasor-based synchronized measurement \cite{synchro} are methods which estimate dynamic changes in frequency. 
Among all of these methods, the phasor-based techniques appear to be more accurate. These methods follow a common way of estimating the positive-sequence phasor at off-nominal frequency, when the DFT is taken at nominal frequency. The phase angle change in the phasor is a function of deviation of frequency from nominal frequency. For a frequency modulated signal $x(t)=Acos(\phi(t))$, where $A$ represents the magnitude and $\phi$ represents the cumulative phase angles, instantaneous frequency is calculated as $f(t) = (1/2\pi)(d\phi /dt)$.
In the following paper, we propose a novel method which takes the advantage of phasor-based technique to estimate amplitude of the original voltage signal and its moving average signal and hence the deviation in frequency. This method has advantage in terms of accuracy, speed and robustness over other methods mentioned in the paper. We first present a theoretical analysis of the proposed method and then later support it with simulation results for a clean 3-phase signal and real signal from generator and a power system.